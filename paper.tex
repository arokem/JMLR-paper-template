\documentclass[twoside,11pt]{article}

% Any additional packages needed should be included after jmlr2e.
% Note that jmlr2e.sty includes epsfig, amssymb, natbib and graphicx,
% and defines many common macros, such as 'proof' and 'example'.
%
% It also sets the bibliographystyle to plainnat; for more information on
% natbib citation styles, see the natbib documentation, a copy of which
% is archived at http://www.jmlr.org/format/natbib.pdf

\usepackage{jmlr2e}

% Definitions of handy macros can go here

\newcommand{\dataset}{{\cal D}}
\newcommand{\fracpartial}[2]{\frac{\partial #1}{\partial  #2}}

% Heading arguments are {volume}{year}{pages}{submitted}{published}{author-full-names}

% Short headings should be running head and authors last names
\ShortHeadings{A Really Awesome MLHC Article}{Lastname, PhD and Lastname, MD}
\firstpageno{1}

\begin{document}

\title{A Really Awesome MLHC Article}

\author{\name Firstname Lastname \email name@email.edu \\
       \addr Department of Health Research\\
       University of Awesome\\
       Some Fair City, State, Country 
       \AND
       \name Firstname Lastname \email name@email.edu \\
       \addr Department of Health Research\\
       University of Other Kinds of Awesome\\
       Some Fair City, State, Country} 

\maketitle

\begin{abstract}
  Summary of the article, a short paragraph will do---doesn't have to
  be in a structured format.  
\end{abstract}

\section{Introduction}

Tells us a bit about the problem.  Recent advances in machine learning
\citep{cite1} have resulted in great things happening in healthcare.
In particular, \citet{cite2} describes a spiffy technique to save even
more lives.  In this work, we...

As you talk away, keep in mind that MLHC papers are meant to be read
by computer scientists and clinicians.  In the later sections, you
might have to use some medical terminology that a computer scientist
may not be familiar with, and you might have to use some math that a
clinician might not be familiar with.  That's okay, as long as you've
done your best to make sure that the core ideas can be understood by
an informed reader in either community.  And certainly this
introduction should be readable by all! 

\section{Cohort}

Describe the cohort.  Give us the details of any inclusion/exclusion
criteria, what data were extracted, how features were processed,
etc. In fact, you probably will want headings for

\subsection{Cohort Selection} 
with choice of criteria and basic numbers, as well as any relevant
information about the study design (such how cases and controls were
identified, if appropriate), 

\subsection{Data Extraction} 
with what raw information you extracted or collected, including any
assumptions and imputation that may have been used, and 

\subsection{Feature Choices} 
with how you might have converted the raw data into features that were
used in your algorithm. 

The goal is to provide enough detail so that someone could replicate
the study.  For more clinical application papers, each of the sections
above might be several paragraphs because we really want to understand
the setting.  

For the submission, please do \emph{not} include the name of the
institutions for any private data sources.  However, in the
camera-ready, you may include identifying information about the
institution as well as should include any relevant IRB approval
statements.  

\section{Methods}

Tell us your techniques!  If your paper is develops a novel machine
learning method or extension, then be sure to give the technical
details---as you would for a machine learning publication---here and,
as needed, in appendices.  If your paper is developing new methods,
this section might be several pages.  

If you are combining existing methods, then you don't need to provide
a ton of detail: feel free to just cite other packages and papers and
tell us how you put them together.  

\section{Results} 

\subsection{Evaluation Approach/Study Design} 

Before jumping into the results: what exactly are you evaluating?
Tell us (or remind us) about your study design and evaluation
criteria.  

\subsection{Results on Application A} 

Give us some numbers about how well your awesomeness works, especially
in comparison to some baselines.  You should provide a summary of the
results in the text, as well as in tables (such as
table~\ref{tab:example}) and figures (such as figure~\ref{fig:example}).  

You may use subfigures/wrapfigures so that figures don't have to span
the whole page or multiple figures are side by side.

\begin{table}[htbp]
  \centering 
  \caption{These are our results.} 
  \begin{tabular}{|l|l|}\hline
    Us & Top Score \\ \hline
    Baseline & Less Impressive Score \\ \hline 
  \end{tabular}
  \label{tab:example} 
\end{table}

\begin{figure}[htbp]
  \centering 
  \includegraphics[width=1.5in]{smile.jpeg} 
  \caption{Feeling awesome.}
  \label{fig:example} 
\end{figure} 

\subsection{Results on Application B} 

Did more than one experiment type?  Sweet! 

\section{Discussion and Related Work} 

This is where you give us some technical understanding about why your
awesomeness is awesome, and maybe even some times when it isn't so we
know when we should be using it.  Discuss both technical and clinical
implications, as appropriate.  

Make sure you also put your awesomeness in the context of related
work.  Who else has worked on this problem, and how did they approach
it?  What makes your direction interesting or distinct?

% ACKNOWLEDGEMENTS ONLY GO IN THE CAMERA-READY, NOT THE SUBMISSION
% \acks{Many thanks to all collaborators and funders!}

\bibliography{paper}


\appendix
\section*{Appendix A.}

Some more details about those methods, so we can actually reproduce
them.  After the blind review period, you could link to a repository
for the code also ;) 

\end{document}
